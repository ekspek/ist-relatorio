% arara: xelatex: { shell: yes, synctex: yes }
\RequirePackage[l2tabu, orthodox]{nag}
\documentclass[palatino,english]{ist-report}

% == BEGIN PREAMBLE ==
% Packages and configurations here aren't really necessary to define the style,
% but are recommended to use. Some packages require users to know extra commands
% to use properly, so I won't include them in the class file for now.

% -- Code snippets (this one is a challenge to implement)
\usepackage{minted}
\definecolor{bg}{rgb}{0.95,0.95,0.95}
\setminted[c]{linenos, bgcolor = bg, breaklines}
\setmintedinline[c]{bgcolor = {}} 

% -- Bibliography
\usepackage{csquotes}
\usepackage{fvextra}
\usepackage{biblatex}
\addbibresource{doc.bib}

% -- Extra math options
\usepackage{siunitx}

% -- Extra symbols
\usepackage{amssymb}
\usepackage{textcomp}
\usepackage{gensymb}

% --  Image and float settings
\usepackage{graphicx}
\graphicspath{{graphics/}}
\usepackage{subcaption}

% -- Graphs and diagrams
\usepackage{tikz}
\usepackage{pgfplots}
\usetikzlibrary{arrows.meta,positioning}
\pgfplotsset{compat = 1.5, table/search path = {data/}}
% == END PREAMBLE ==

\newrobustcmd*{\package}[1]{\texttt{#1}}

\begin{document}

\thispagestyle{empty}

\title{Report Example}
\author{Daniel de Schiffart}
\date{2018}
\maketitle{}

{\hypersetup{linkcolor = black} \tableofcontents}

\begin{abstract}
	The \LaTeX{} class developed in this project was developed as a template for reports developed for the Instituto Superior Técnico of Universidade de Lisboa, themed around the style the university has developed for its own internal documentation, complemented by using specifications defined by the university itself, all the while taking some creative liberties with the missing definitions.
	
	This document details the class file, what it exactly contains, what its options are, and a concise documentation of its development, which will grow as the class file progresses. The class file itself, \texttt{ist-report.cls}, should be in the same directory as the \texttt{doc.tex} file.
\end{abstract}

\section{Introduction}

This project has the focus on creating a generic \LaTeX{} report example for IST students to base their own reports upon, to streamline much of the typesetting end of things for an IST report so students and other users can focus on writing content instead of spending time getting \LaTeX{} to work. Hopefully it can also work as a first dive for those who want to learn \LaTeX{} a tad more deeply than that. The final goal being having a working example using a custom class file that contains as many definitions as it can with the least possible amount of custom packages, with a couple of simple customisation options, so it can work as a simple template for those who need it. Obviously I'm taking some liberties with how it will end up looking, but hopefully it'll look pretty and professional and presentable and not really as over-the-top as it'd look like if you'd let me loose with Illustrator and a bottle of Jägermeister.

This project also aims to be a learning experience for me. I've only scratched the surface of \LaTeX{} apparently, and looking at examples from other universities and multiple other online resources made me want to see where I can go before giving up or losing my head.

A lot of information was acquired from appendix A of \textit{The \LaTeX{} Companion} \cite{latex-companion}, which provided a solid base on how to build a class file based on the needs at hand. I'll keep documenting most other resources as they come up throughout development.

\section{Class File Documentation}

\subsection{Options}

\begin{description}
	\item [Compiler detection] The class detects the compiler being used and will react accordingly, defining certain settings to support the compiler's capabilities and shortcomings. Preferred compiler is \XeLaTeX{}, to make full use of the class file\footnotemark{}. \footnotetext{Actually, for the time being the class makes no special use of unicode compilers (except removing font encoding). Future plans are to add Arial and other OpenType fonts to the mix though.}
	\item [palatino] This option changes the main font to the palatino-inspired \package{newpxtext}, with math text using \package{eulervm} and typewriter text to \package{nimbusmono}. Off by default.
\end{description}

\subsection{Packages}

I'm still settling on a way to represent packages, descriptions and links. Until then, they're all listed here in an \texttt{itemize} environment.
\begin{itemize}
	\item \package{etoolbox}
	\item \package{mathtools}
	\item \package{polyglossia [portuguese, english]}
	\item \package{geometry}
	\item \package{graphicx}
	\item \package{hyperref}
	\item \package{fontspec}
	\item \package{microtype}
	\item \package{lmodern}
	\item \package{xcolor}
	\item \package{fancyhdr}
	\item \package{footmisc [bottom]}
	\item \package{caption}
	\item \package{metalogo}
	\item \package{helvet}
	\item \package{tikz}
\end{itemize}

\subsection{Class Contents}

The class file is based off of the \LaTeX{} basic \package{article} class, edited to fit the Instituto Superior Técnico theme.

\subsection{Changelog}

\begin{description}
	\item [V0.1.0] First working release. Included header and footer, \SI{2.5}{\centi\meter} margins. Implemented \package{palatino} option to separate font choices (default is \package{lmodern}), implemented \TeX{} conditionals to detect \XeTeX{} or pdf\TeX{}.
	\item [V0.2.0] Added \package{portuguese} and \package{english} options to change report macro language, included IST colors and applied them to \package{hyperref}. Option \package{portuguese} runs by default and doesn't need to be included.
	\item [V0.2.1] Added black and white option (except logos), added IST logo crop margins.
	\item [V0.3.0] Reworked \package{twoside} option to work with already implemented features. Implemented \package{etoolbox} package for \TeX{} conditionals.
	\item [V0.4.0] Added the \verb|\makecover| command with a simple cover example. Added the first version of the placeholder logos.
	\item [V0.5.0] Separated covers and header and footer styles into separate files (might undo later for release). Headers and footers are now bound to the \verb|\pagestyle| command. By default, the \package{default} page style is loaded. Fixed wrong IST logo crops.
	\item [V0.5.1] Added \package{basic} option. Added one more style.
	\item [V0.6.0] Added \package{purist} option, only compatible with non-unicode engines.
	\item [V0.6.1] First working version of example cover. Fixed cover to use the \package{article} class \verb|\author{}| command. Changed default style. Fixed group number command and implemented it into cover. Small code cleanup.
	\item [V0.6.2] Removed \package{basic} option.
\end{description}

\pagebreak

\printbibliography

\end{document}
